%! TeX encoding = UTF-8 unicode
%%%%%%%%%%%%%%%%%%%%%%%%%%%%%%%

\documentclass[10pt,aps,prb,amsmath,amssymb,twocolumn,letterpaper,nobalancelastpage,final,citeautoscript,floatfix,raggedbottom,superscriptaddress]{revtex4-1}

%%%%%%%%%%%%%%%
% MY SETTINGS
%%%%%%%%%%%%%%%v
\usepackage[usenames,dvipsnames]{color}
\usepackage{graphicx}
\usepackage{microtype}
\usepackage[bookmarks=false,colorlinks]{hyperref}
\usepackage{xfrac}
\hypersetup{
    linkcolor=RubineRed,          % color of internal links
    citecolor=ForestGreen,        % color of links to bibliography
    filecolor=Mulberry,      % color of file links
    urlcolor=RoyalBlue           % color of external links
}
\usepackage{natbib}
%additional commands
\newcommand{\textapprox}{\raisebox{0.5ex}{\texttildelow}}
\newcommand{\gvs}{GaV$_4$S$_8$}
\newcommand{\gvse}{GaV$_4$Se$_8$}
\newcommand{\gms}{GaMo$_4$S$_8$}
\newcommand{\gmse}{GaMo$_4$Se$_8$}
\newcommand{\gns}{GaNb$_4$S$_8$}
\newcommand{\gnse}{GaNb$_4$Se$_8$}
\newcommand{\gts}{GaTa$_4$S$_8$}
\newcommand{\gtse}{GaTa$_4$Se$_8$}

\newcommand{\dpg}[1]{\textcolor{red}{#1}}


%%%%%%%%%%%%%
% MAIN TEXT
%%%%%%%%%%%%%

\begin{document}

\title{Analysis of alloy steel composition-property relationship \\ using machine learning:  beyond human physical intuition}

\author{Raymond Wang}
  \affiliation{Department of Materials Science and Engineering, Northwestern University, Evanston, Illinois  60208, USA}

  
%%%%%%%%%%%%%
% ABSTRACT
%%%%%%%%%%%%%

\begin{abstract}
This is the abstract, write it at last.
\end{abstract}

\maketitle

%%%%%%%%%%%%%%%%%%%%%%%%%%%%%%%%%%%%%%
% MOTIVATION
%%%%%%%%%%%%%%%%%%%%%%%%%%%%%%%%%%%%%%

\section{Motivation}
why are you doing this?
flowchart

%%%%%%%%%%%%%%%%%%%%%%%%%%%%%%%%%%%%%%
% DATA ACQUISITION
%%%%%%%%%%%%%%%%%%%%%%%%%%%%%%%%%%%%%%
\section{data acquisition}
how and where did you get it. how did you deal with blocking problems.

portable code
%%%%%%%%%%%%%%%%%%%%%%%%%%%%%%%%%%%%%%
% FEATURIZATION
%%%%%%%%%%%%%%%%%%%%%%%%%%%%%%%%%%%%%%
\section{featurization}

What is the format of raw data, how did you collect them to one dataframe, how to convert them to float64, why pick these elements and properties (number count).

%%%%%%%%%%%%%%%%%%%%%%%%%%%%%%%%%%%%%%%%%
% DATA ANALYSIS WITH HUMAN INTELLIGENCE
%%%%%%%%%%%%%%%%%%%%%%%%%%%%%%%%%%%%%%%%%
\section{data analysis with human intelligence}

% experience

Carbon: increase tensile strength and hardness; decrease ductility, toughness.

Sulfur: improve machinability; decrease weldability; impact toughness and ductility.


Silicon: increase tensile/yield strength, hardness, forgeability and magnetic permeability; deoxidier and degasifier.

Phosphorous: increase strength and hardness; improve machinability and corrosion resistance.

Manganese: increase tensile strength, hardness and reduce brittleness; deoxider and degasifier.

Chromium: increase tensile strength, hardness and corrosion resistance.

Nickel: increase strength, hardness, toughness and resistance to corrosion.

Molybdenum: increase toughness, strength, hardness, machinability, resistance to corrosion.

Copper: increase corrosion resistance.

Density
Hardness, Vickers:
Thermal conductivity:
Specific heat capacity:
CTE, linear:
Electrical resistivity:
Elongation at break:
Bulk modulus:
Modulus of elasticity:
Shear modulus:
Poissons ratio:
Tensile strength, yield:
Tensile strength, break:


% analysis
Iron: decrease thermal conductivity, decrease specific heat capacity, slightly increase modulus of elasticity, slightly decrease CTE

Carbon: decrease thermal conductivity, slightly increase specific heat capacity, increase electrical resistivity, decrease shear modulus

Sulfur: increase specific heat capacity, slightly increase electrical resistivity

Silicon: decrease density, slightly increase specific heat capacity and CTE, slightly increase electrical resistivity

Phosphorous: decrease thermal conductivity, slightly increase specific heat capacity, increase electrical resistivity, slightly increase modulus of elasticity

Manganese: slightly increase specific heat capacity, slightly increase electrical resistivity, slightly decrease shear modulus

Chromium: decrease density, increase thermal conductivity, slightly increase CTE, slightly decrease modulus of elasticity

Nickel: increase thermal conductivity, slightly increase specific heat capacity and CTE, slightly decrease modulus of elasticity, slightly increase tensile strength

Molybdenum: increase specific heat capacity, increase electrical resistivity

Copper: increase specific heat capacity, increase electrical resistivity

Discuss influence of element to properties, discuss their correlation, whether it is expected/intuitive, find some weakly correlated target props

%%%%%%%%%%%%%%%%%%%%%%%%%%%%%%%%%%%%%%%%%%%%%%
% DATA ANALYSIS WITH ARTIFICIAL INTELLIGENCE
%%%%%%%%%%%%%%%%%%%%%%%%%%%%%%%%%%%%%%%%%%%%%%
\section{data analysis with artificial intelligence}

CTE linear is dropped for more instances to learn.
\subsection{NEED NEW NAME}
benchmark (x-axis: algo, y-axis: log(r2+1), dataset: predict x from the rest dataset)

table of hyper-parameters

\subsection{predicting composition-property relationship}
plot (single algo, x-axis: element, y-axis: log(r2+1))
Overfitting, plot learning curve

use XGBoost

%%%%%%%%%%%%%%%%%%%%%%%%%%%%%%%%%%%%
% CONCLUSIONS
%%%%%%%%%%%%%%%%%%%%%%%%%%%%%%%%%%%
\section{conclusions}

%\bibliography{reference.bib}

\end{document}



